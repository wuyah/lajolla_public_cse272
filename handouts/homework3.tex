\documentclass{article}
\usepackage{hyperref}
\usepackage{listings}
\usepackage{color}
\usepackage{geometry}
\usepackage{graphicx}
\usepackage{amsmath}
\usepackage{caption}
\usepackage{subcaption}
\geometry{margin=1in}
\pdfminorversion=6

\newcommand\TODO[1]{\textcolor{red}{TODO: #1}}

\newcommand\header[2]{
    \begin{center}
        {\large
        UCSD CSE 272 Assignment #1: \\
        \vspace{0.3cm}
        \Large
        #2}
    \end{center}
}

\definecolor{dkgreen}{rgb}{0,0.6,0}
\definecolor{gray}{rgb}{0.5,0.5,0.5}
\definecolor{mauve}{rgb}{0.58,0,0.82}
\lstset{frame=tb,
        aboveskip=3mm,
        belowskip=3mm,
        showstringspaces=false,
        columns=flexible,
        basicstyle={\small\ttfamily},
        numbers=none,
        numberstyle=\tiny\color{gray},
        keywordstyle=\color{blue},
        commentstyle=\color{dkgreen},
        stringstyle=\color{mauve},
        breaklines=true,
        breakatwhitespace=true,
        tabsize=2
}


\begin{document}

\header{3}{Paper review}

In this homework, you will read a paper about rendering and write a review about it. The goal is to learn how to think about academic work critically and communicate clearly. The papers you choose can be the ones we talked about in the class or anything that is rendering related.\footnote{Ke-Sen Huang maintains a useful \href{http://kesen.realtimerendering.com/}{webpage} of papers published at different graphics conferences.} Talk to us if you need any suggestions.

Write your review in the format of the \href{https://sa2020.siggraph.org/en/submissions/technical-papers/review-form}{SIGGRAPH review form}. It needs to contain the following sections:
\begin{itemize}
	\item \textbf{Description}. Summarize the paper in a few sentences, and assess the magnitude of contributions.
	\item \textbf{Clarity of exposition}. Is the presentation of the paper clear? If not, how can they be improved?
	\item \textbf{Quality of references}. Are all related references cited? List additional references if needed.
	\item \textbf{Reproducibility}. Could the work be reproduced from the information of the paper alone? Are the limitations and drawbacks clear?
	\item \textbf{Recommendation}. Rate the paper with the following: \emph{strong reject}, \emph{reject}, \emph{borderline reject}, \emph{borderline accept}, \emph{accept}, \emph{strong accept}.
	\item \textbf{Explanation of rating}. This is the most important part of your review. Explain your rating by discussing the strengths and weaknesses of the submission, in terms of the contributions and potential impacts. Include suggestions for improvements. Be thorough, be fair, and be courteous.
\end{itemize}

Submit the review in text to Canvas.

Here are some tips from the SIGGRAPH \href{https://sa2020.siggraph.org/en/submissions/technical-papers/reviewer-instructions}{website}:

\textit{Look for what's good or stimulating in the paper. Minor flaws can be corrected and shouldn't be a reason to reject a paper. Each paper that is accepted must, however, be technically sound and make a substantial contribution to the field.}

\textit{Please be specific and detailed in your reviews. In the discussion of related work and references, simply saying "this is well known" or "this has been common practice in the industry for years" is not sufficient: please cite specific publications or public disclosures of techniques! The Explanation section is easily the most important of the review. Your discussion, sometimes more than your score, will help the Papers Committee decide which papers to accept, so please be thorough. Your reviews will be returned to the authors, so you should include any specific feedback on ways the authors can improve their papers.}

Resources and tips for writing reviews
\begin{itemize}
	\item \href{http://people.csail.mit.edu/fredo/review.pdf}{Fredo Durand's slides on peer review}
	\item \href{https://twitter.com/nmwsharp/status/1491528553766305801}{Tips from Nick Sharp}
	\item \href{https://graphics.stanford.edu/~kayvonf/notes/systemspaper/}{Kayvon Fatahalian on what constitutes a research contribution}
	\item \href{https://www.natolambert.com/guides/how-to-review-a-paper}{How to review a paper by Nato Lambert}
	\item \href{https://web.archive.org/web/20180124233323/http:/www.siggraph.org/s2008/submissions/juried/papers/review_writing.php}{Greg Turk's advices}
	\item \href{https://www.siggraph.org/sites/default/files/kajiya.pdf}{Jim Kajiya on the SIGGRAPH review process}
	\item \href{https://sigmodrecord.org/publications/sigmodRecord/0812/p100.open.cormode.pdf}{How not to review a paper by Graham Cormode}
	\item \href{https://www.cs.princeton.edu/~jrex/teaching/spring2005/fft/reviewing.html}{The task of the referee by Alan Jay Smith}
	\item \href{https://perceiving-systems.blog/en/news/novelty-in-science}{Michael Black on novelty}
\end{itemize}

\end{document}